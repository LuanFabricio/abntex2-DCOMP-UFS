\chapter{Introdução}

Redes Neurais Artificiais (ANN) são ferramentas poderosas para auxiliar a sociedade.
Podendo ser utilizadas em diversas tarefas, como reconhecimento facial, onde a rede é treinada para realizar a
classificação da face da pessoa, permitindo que ela seja usada em várias áreas diferentes, indo de entretenimento até
segurança.

\section{Motivação}
O uso de Redes Neurais Artificiais vem crescendo bastante no ramo de computação visual, principalmente desde 2012,
quando Redes Neurais Convolucionais (CNN) começaram a ser utilizadas para classificação de imagens \cite{alexnet}.
Um dos principais usos desse tipo de rede é na detecção de face, que é muito relevante para a área de segurança e
vigilância, onde o modelo pode fazer a detecção do rosto de uma pessoa, abrir uma porta ou enviar uma notificação
para algum segurança. Porém, essa abordagem necessita de uma quantidade elevada de poder computacional, tornando
inviável que tal tipo de produto tenha um baixo tempo de resposta, o que pode atrapalhar a experiência do usuário, ou
reduz a efetividade da ação que será tomada.

Um dos principais problemas das Redes Neurais Profundas, como CNN, é que elas necessitam de um alto processamento e uso
de memória, o que acaba dificultando a sua execução em dispositivos com poder computacional e memória limitados (como os dispositivos embarcados).
Porém, existem técnicas que podem ser aplicadas para reduzir o poder computacional necessário, como o uso de
destilação de conhecimento \cite{hinton2015distilling}, poda e quantização.
Possibilitando a implantação do modelo em sistemas embarcados na borda, de forma que a latência dispositivo seja baixa.

\section{Objetivo principal}

O objetivo deste trabalho é desenvolver uma Rede Neural Convolucional que consiga realizar a tarefa de reconhecimento
facial em microcontroladores, na borda.
De forma que, o reconhecimento facial seja realizado dentro do dispositivo embarcado, permitindo que ele realize essa
tarefa e execute uma rotina, com a menor latência possível, por exemplo, desbloqueando uma porta assim que o rosto for
reconhecido pelo sistema.
Para que ele seja alcançado, será necessário comprimir um modelo base, para que ele exija pouco poder computacional
e uso de memória.

\section{Metodologia}
Para atingir o objetivo do estudo, foi necessário dividir o processo em algumas etapas, cada uma sendo
essencial para que o objetivo do trabalho seja atingido. Sendo elas:

\begin{enumerate}
	\item \textbf{Levantamento do estado da arte:}

		Nessa etapa, são selecionados artigos que possuem o objetivo parecido com o deste artigo,
		com base nesses artigos serão testadas novas técnicas para compressão de modelos.

	\item \textbf{Reprodução do estado da arte:}

	\begin{enumerate}
		\item \textbf{Seleção de base e treino de modelos:}

			Nesta é selecionada uma base de dados e a partir dela serão desenvolvidos modelos,
			com o objetivo de atingir uma alta acurácia, sem sofrer \textit{overfitting}.

		\item \textbf{Aplicação de técnicas de compressão para Modelos:}

			Após definir e treinar os modelos, serão aplicadas técnicas de compressão, tendo como
			objetivo ter uma acurácia parecida com a do modelo original. Onde as técnicas aplicadas
			foram: poda, quantização e destilação de conhecimento.

		\item \textbf{Avaliação do desempenho:}

			Depois de treinar e aplicar técnicas de compressão, os dados dos modelos serão coletados e
			avaliados. Para realizar essa avaliação, será necessário utilizar um conjunto de testes.

		\item \textbf{Análise e comparação dos resultados:}

			Para finalizar, os dados dos modelos serão comparados e analisados. Com o objetivo de
			identificar o melhor modelo e descobrir quais foram os motivos para que esse modelo tenha se
			saído tão bem, mesmo após a aplicação de compressão. Nesta etapa as métricas de acurácia e
			tamanho do modelo são avaliadas.
	\end{enumerate}

	% TODO: Terminar
	\item \textbf{Escolha do modelo para reconhecimento facial:}

	\item \textbf{Compressão e implantação do modelo em hardware limitado:}

	\item \textbf{Desenvolvimento do estudo de caso:}
\end{enumerate}

Onde as etapas 3, 4 e 5 serão realizadas no Trabalho de Conclusão de Curso 2.

\section{Estrutura do documento}
Este documento foi dividido em capítulos, onde cada um apresenta uma proposta diferente:
\begin{itemize}
	\item Capítulo 2 - \textbf{Conceitos Básicos}: Apresenta os tópicos principais para o entendimento
		do trabalho.
	\item Capítulo 3 - \textbf{Trabalhos Relacionados}: Apresenta uma revisão dos trabalhos
		relacionados ao tema do trabalho.
	\item Capítulo 4 - \textbf{Resultados Preliminares}: Apresenta os resultados preliminares dos
		experimentos realizados durante o trabalho.
	\item Capítulo 5 - \textbf{Planos de continuidade}: Contém o planejamento da continuidade do
		Trabalho de Conclusão 2.
\end{itemize}
