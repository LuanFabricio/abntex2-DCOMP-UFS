\chapter{Introdução}

% TODO: Melhorar e aprofundar sobre ANN
Redes Neurais Artificiais, ou \textit{Artificial Neural Network} (ANN), são ferramentas poderosas para auxiliar a
sociedade.
Podendo ser utilizadas em diversas tarefas, como reconhecimento facial, onde a rede é treinada para realizar a
classificação da face da pessoa, permitindo que ela seja usada em várias áreas diferentes, indo de entretenimento até
segurança.

\section{Motivação}
% TODO: Talvez seja interessante rescrever para exaltar que o
% motivo foi o artigo
O uso de Redes Neurais Artificiais vem crescendo bastante no ramo de computação visual, principalmente desde 2012,
quando Redes Neurais Convolucionais ou \textit{Convolutional Neural Networks} (CNN) começaram a ser utilizadas para
classificação de imagens \cite{alexnet}.
Um dos usos desse tipo de rede é na detecção de face, que é muito relevante para a área de segurança e
vigilância, onde o modelo pode fazer a detecção do rosto de uma pessoa, abrir uma porta ou enviar uma notificação
para algum segurança.
Porém, essa abordagem necessita de uma quantidade elevada de poder computacional, tornando inviável que tal tipo de
produto seja embarcado e mantenha um baixo tempo de resposta, o que pode atrapalhar a experiência do usuário, ou
reduzir a efetividade da ação que será tomada.

Um dos principais problemas das Redes Neurais Profundas, como CNN, é que elas necessitam de um alto processamento e uso
de memória, o que acaba dificultando a sua execução em dispositivos com poder computacional e memória limitados (como os dispositivos embarcados).
Porém, existem técnicas que podem ser aplicadas para reduzir o poder computacional necessário, como o uso de
destilação de conhecimento \cite{hinton2015distilling}, poda e quantização.
Possibilitando a implantação do modelo em sistemas embarcados na borda, de forma que a latência do dispositivo seja
baixa.

\section{Objetivos}

O objetivo deste trabalho é comprimir uma Rede Neural Convolucional, permitindo que ela realize o reconhecimento
facial em microcontroladores, na borda.
Nele também serão tratadas formas de comprimir e otimizar o modelo, para que a sua versão final consiga ser embarcada
em um dispositivo com hardware limitado, mantendo acurácia alta e baixa latência.

\subsection{Objetivo específicos}
O trabalho estará completo se os seguintes objetivos forem alcançados:

\begin{itemize}
	% TODO: Reescrever, deixando claro que o objetivo é embarcar o modelo.
	% Deixar claro que é necessário que o modelo comprimido também tenha uma alta acurácia, F1-score etc.
	% Deixar claro que é necessário que o modelo comprimido não utilize poucos recursos computacionais
	%	(armazenamento, memória e uso de cpu) e que a latência também seja baixa.
	\item Aplicar técnicas de compressão e otimização em um modelo, assim reduzindo o uso de CPU e memória,
		o permitindo que seja embarcado.
	% TODO: Rever
	\item Validar performace do modelo, para garantir que a acurácia e F1-score do modelo seja mantida.
	\item Embarcar o modelo comprimido de forma que ele consiga realizar operações na borda, mantendo acurácia
		alta e baixa latência.
\end{itemize}

\section{Metodologia}
Para atingir o objetivo do estudo, foi necessário dividir o processo em algumas etapas, cada uma sendo
essencial para que o objetivo do trabalho fosse atingido. Sendo elas:

\begin{enumerate}
	\item \textbf{Levantamento do estado da arte:}

		Nessa etapa, são selecionados artigos que possuem o objetivo similar ao deste artigo,
		com base nesses artigos serão testadas novas técnicas para compressão de modelos.

	\item \textbf{Reprodução do estado da arte:}
		% TODO: Explicar melhor o que foi feito nesta etapa.
		% Verificar se ela irá aparecer na etapa final do trabalho.
		% TODO: Revisar

	\begin{enumerate}
		\item \textbf{Seleção de base e treino de modelos:}

			Nesta etapa, uma base de dados é selecionada e a partir dela serão desenvolvidos modelos,
			com o objetivo de atingir uma alta acurácia, sem sofrer \textit{overfitting}.

		\item \textbf{Aplicação de técnicas de compressão para Modelos:}

			Após definir e treinar os modelos, serão aplicadas técnicas de compressão, tendo como
			objetivo ter uma acurácia parecida com a do modelo original. Onde as técnicas aplicadas
			foram: poda, quantização e destilação de conhecimento.

		\item \textbf{Avaliação do desempenho:}

			Depois de treinar e aplicar técnicas de compressão, os dados dos modelos serão coletados e
			avaliados. Para realizar essa avaliação, será necessário utilizar um conjunto de testes.

		\item \textbf{Análise e comparação dos resultados:}

			Para finalizar, os dados dos modelos serão comparados e analisados. Com o objetivo de
			identificar o melhor modelo e descobrir quais foram os motivos para que esse modelo tenha se
			saído melhor, mesmo após a aplicação de compressão. Nesta etapa as métricas de acurácia e
			tamanho do modelo são avaliadas.
	\end{enumerate}

	\item \textbf{Escolha do modelo para reconhecimento facial:}

		Nesta etapa serão avaliados os modelos com base em métricas como acurácia, F1 score, latência e ocupação de
		memória. Após a avaliação será escolhido um modelo que servirá como base nas próximas etapas.

	\item \textbf{Compressão e implantação do modelo em hardware limitado:}

		Depois de escolher um modelo de reconhecimento facial, ele será comprimido de forma que possa ser executado
		em sistemas embarcados, mantendo a acurácia alta e latência baixa.

	\item \textbf{Desenvolvimento do estudo de caso:}

		Com o modelo final comprimido ao ponto de ser implantado em um sistema embarcado, será desenvolvida uma aplicação 	 	 que servirá como experimento para avaliar a eficácia do modelo dentro de dispositivos com hardware limitado.

\end{enumerate}

As etapas 3, 4 e 5 serão realizadas no Trabalho de Conclusão de Curso 2.

\section{Estrutura do documento}
Este documento foi dividido em capítulos, onde cada um apresenta uma proposta diferente:
\begin{itemize}
	\item Capítulo 2 - \textbf{Conceitos Básicos}: Apresenta os tópicos principais para o entendimento
		do trabalho.
	\item Capítulo 3 - \textbf{Trabalhos Relacionados}: Apresenta uma revisão dos trabalhos
		relacionados ao tema do trabalho.
	\item Capítulo 4 - \textbf{Resultados Preliminares}: Apresenta os resultados preliminares dos
		experimentos realizados durante o trabalho.
	\item Capítulo 5 - \textbf{Planos de continuidade}: Contém o planejamento da continuidade do
		Trabalho de Conclusão 2.
\end{itemize}
