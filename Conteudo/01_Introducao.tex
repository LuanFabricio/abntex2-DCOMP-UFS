\chapter{Introdução}

\section{Motivação}
O uso de Redes Neurais Artificiais vem crescendo bastante no ramo de computação visual, principalmente
desde 2012, quando Redes Neurais Convolucionais começaram a ser utilizadas para classificação de imagens
\cite{alexnet}. Entretanto

% TODO: Escrever motivação

\section{Objetivo principal}
% TODO: Escrever objetivo principal
O objetivo principal do projeto é comprimir uma Rede Neural que calcula a velocidade de veículos. Para fazer isso será
necessário utilizar uma série de métodos e técnicas de compressão, como destilamento de conhecimento, poda e quantização de Redes Neurais.

\section{Metodologia}
Para atingir o objetivo do estudo, foi necessário dividir o processo em algumas etapas, cada uma sendo
essencial para que o objetivo do trabalho seja atingido. Sendo elas:

\begin{enumerate}
	\item \textbf{Seleção de artigos selecionados:}

		Nessa etapa, são selecionados artigos que possuem o objetivo parecido com o deste artigo,
		com base nesses artigos serão testadas novas técnicas para compressão de modelos.

	\item \textbf{Seleção de base e treino de modelos:}

		Nesta é selecionada uma base de dados e a partir dela serão desenvolvidos modelos,
		com o objetivo de atingir uma alta acurácia, sem sofrer \textit{overfitting}.

	\item \textbf{Aplicação de técnicas de compressão para Modelos:}

		Após definir e treinar os modelos, serão aplicadas técnicas de compressão, tendo como
		objetivo ter uma acurácia parecida com a do modelo original. Onde as técnicas aplicadas
		foram: poda, quantização e destilamento de conhecimento.

	\item \textbf{Avaliação do desempenho:}

		Depois de treinar e aplicar técnicas de compressão, os dados dos modelos serão coletados e
		avaliados. Para realizar essa avaliação, será necessário utilizar um conjunto de teste.

	\item \textbf{Análise e comparação dos resultados:}

		Para finalizar, os dados dos modelos serão comparados e analisados. Com o objetivo de
		identificar o melhor modelo e descobrir quais foram os motivos para que esse modelo tenha se
		saído tão bem, mesmo após a aplicação de compressão. Nesta etapa as métricas de acurácia e
		tamanho do modelo são avaliadas.

\end{enumerate}

\section{Estrutura do documento}
Este documento foi divido em capítulos, onde cada um apresenta uma proposta diferente:
\begin{itemize}
	\item Capitulo 2 - \textbf{Conceitos Básicos}: Apresenta os tópicos principais para o entendimento
		do trabalho.
	\item Capítulo 3 - \textbf{Trabalhos Relacionados}: Apresenta uma revisão dos trabalhos
		relacionados ao tema do trabalho.
	\item Capítulo 4 - \textbf{Resultados Preliminares}: Apresenta os resultados preliminares dos
		experimentos realizados durante o trabalho.
	\item Capítulo 5 - \textbf{Planos de continuidade}: Contém o planejamento da continuidade do
		Trabalho de Conclusão 2.
\end{itemize}
