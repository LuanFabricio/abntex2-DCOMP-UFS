\chapter{Trabalhos relacionados}\label{cap_trabalhos_relacionados}

Neste capítulo serão discutidos os trabalhos relacionados à compressão de CNNs com foco em dispositivos
embarcados.

\section{Trabalhos acadêmicos}

Os seguintes critérios de busca foram utilizados para filtrar os trabalhos acadêmicos:

\begin{itemize}
	\item Os artigos devem ser relacionado ao tema de CNNs com compressão para sistemas embarcados.
	\item Trabalhos publicados entre 2020 e 2023.
	\item Trabalhos escritos em inglês.

	As bases utilizadas foram: IEE Eletronic Library, ACM Digital Library e Science Citation Index Expanded.
	Utilizando a seguinte string de busca:
	\item "CNN"  AND "EMBEDDED" AND "Edge devices" AND ("Pruning" OR "Knowledge distillation" OR "Quantization")
\end{itemize}

\subsection{\textit{A Resource Constrained Pipeline Approach to Embed Convolutional Neural Models (CNNs)}}
O objetivo deste trabalho de dissertação \cite{rafael} é elaborar um modelo de detecção de placas de trânsito que seja
computacionalmente e energeticamente barato. Para atingir esse objetivo, foi elaborada uma pipeline de compressão,
começando pela destilação de conhecimento, e partindo para poda e quantização.

O resultado alcançado foi uma CNN capaz de detectar placas de trânsito, consumindo 59KB de espaço, com $85,91\%$ de
acurácia e F1-Score igual a $85,80\%$, atingindo um tempo de inferência de 80 ms no ESP32 e 83 ms no ESP32-2.
