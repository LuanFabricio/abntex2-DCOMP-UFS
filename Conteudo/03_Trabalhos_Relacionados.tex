\chapter{Trabalhos relacionados}\label{cap_trabalhos_relacionados}

\section{Trabalhos acadêmicos}

\subsection{\textit{A Resource Constrained Pipeline Approach to Embed Convolutional Neural Models (CNNs)}}
O objetivo deste trabalho de dissertação \cite{rafael} é elaborar um modelo de detecção de placas de trânsito que seja
computacionalmente e energeticamente barato. Para atingir esse objetivo, foi elaborada uma pipeline de compressão,
começando pelo destilamento de conhecimento, e partindo para poda e quantização.

O resultado alcançado foi uma CNN capaz de detecta placas de trânsito, consumindo 59KB de espaço, com $85,91\%$ de
acurácia e F1-Score igual a $85,80\%$, atingindo um tempo de inferência de 80 ms no ESP32 e 83 ms no ESP32-2.
