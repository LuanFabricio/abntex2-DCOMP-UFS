\chapter{Experimentos e resultados}

Neste capítulo serão apresentados os experimentos feitos durante o TCC. Eles tiveram como
finalidade avaliar o uso de métodos e técnica de compressão de modelos, focado em dispositivos
embarcados. Nele serão apresentados o dispositivo utilizado (\autoref{sec_dispositivo}),
modelos usados (\autoref{sec_modelo_utilizado}), métodos de treinamento (\autoref{sec_treinamento_modelo})
os métodos de avaliação do modelo (\autoref{sec_avaliacao_modelo}), \textit{dataset} utilizados
(\autoref{sec_datasets}) e resultados (\autoref{sec_resultados}).

\section{Dispositivo utilizado}\label{sec_dispositivo}
O ESP32-S3 N16R8 foi escolhido pelo seu custo-benefício, sendo um aparelho poderoso considerando o seu baixo
custo e consumo energético. Além disso, esse modelo possui 16 MB de memória flash e 8 MB de PSRAM,
permitindo que modelos maiores sejam portados para esse dispositivo sem redução de parâmetros, mesmo que
isso afete o tempo de inferência do modelo.

Para executar os modelos no ESP foi utilizado o modelo person-detection do projeto esp-tflite-micro da Espressif.
\footnote{Disponível em: <https://github.com/espressif/esp-tflite-micro/tree/master> Acessado em Março de 2025}
Além disso, foi necessário adequar os tipos de camada que serão utilizados por cada modelo.


\section{Modelos utilizados}\label{sec_modelo_utilizado}
Para realizar os experimentos, alguns modelos foram utilizados como teste, tanto na etapa de treinamento
(\autoref{sec_treinamento_modelo}) quanto de avaliação (\autoref{sec_avaliacao_modelo}).

\subsection{MobileFaceNet}
% O modelo utilizado foi o MobileFaceNet, pois ele mantém uma acurácia acima de 90\% na tarefa de detecção de faces,
% com tempo de inferência e baixa pegada de memória \cite{leandro}.

O MobileFaceNet é um modelo focado para fazer reconhecimento de faces em tempo real em dispositivos móveis,
assim possuindo uma baixa pegada de memória, abaixo de 5MB e mantendo uma acurácia alta, acima de 90\% \cite{leandro}.

\subsection{Rafael-2}
Rafael-2 é uma variação do modelo Rafael \cite{rafael}, adaptada para realizar a tarefa de reconhecimento facial.
Como o modelo base é simples e focado para dispositivos embarcados, ele possuí uma baixa quantidade de parâmetros,
 o que melhora a performance do modelo em dispositivos embarcados.

\subsection{MobileNetV3Small}
MobileNetV3Small é uma variação do MobileNetV3, com uma quantidade reduzida de parâmetros, reduzindo a sua pegada de
memória e aumentando a performance.

ADICIONAR TABELA COM O CONSUMO DE RECURSOS DO MODELO
\begin{center}
\begin{table}[htb]
\centering
\ABNTEXfontereduzida
\caption[Acurácia dos modelos]{Acurácia dos modelos.}
\label{tabela_acuracia_1}
\begin{tabular}{ |c|c|c|c|c| }
	\hline
	\textbf{Modelo} & \textbf{Parâmetros} & \textbf{Tamanho do arquivo (TFLite)} \\
	\hline
	MobileFaceNet 		& ? 	& 4.5MB \\
	Rafael-2 		& ? 	& 1.4MB \\
	MobileNetV3Small 	& ? 	& 1.4MB \\
	\hline
\end{tabular}
\legend{Fonte: Autor}
\end{table}
\end{center}

\section{Método de treinamento do modelo}\label{sec_treinamento_modelo}
Para fazer o treinamento dos modelos, foi necessário utilizar a técnica de \textit{triplet distillation}
\cite{triplet_distillation_face_recognition},
para que seja possível medir o quão próximo as \textit{embeddings} de uma imagem são parecidas com as de outra.
Essa técnica usa três imagens: \textbf{âncora}, que é serve como imagem base; \textbf{positiva},
que é uma variação da mesma categoria da \textbf{âncora}; e a \textbf{negativa}, que pertence a outra categoria.

No contexto de reconhecimento facial, os \textit{embeddings} das imagens \textbf{âncora} e \textbf{positivas}
devem possuir uma distância de cosseno pequena, enquanto o vetor de característica das imagens
\textbf{âncoras} e \textbf{negativas} devem possuir uma distância de cosseno alta.

\subsection{\textit{Triplet Loss}}
Para realizar o treinamento do modelo MobileFaceNet, foi utilizada a técnica \textit{triplet loss}
\cite{triplet_distillation_face_recognition}. Ela tem como objetivo comparar os \textit{embeddings} de três imagens,
utilizando a distância de cosseno ($D$) de um embedding ($x_i^j$) comparado com outro ($x_i^k$).
%Essa técnica utiliza é descrita pela fórmula \ref{eq_triplet_loss}

\begin{equation}\label{eq_triplet_loss}
	Loss = \frac 1 N \sum _i ^N max(D(x_i^a, x_i^p) - D(x_i^a, x_i^n) + m, 0))
\end{equation}
%
% \begin{equation}\label{eq_triplet_loss_teacher_dist}
% 	d = max(T(x_i^a, x_i^n) - T(x_i^a, x_i^p), 0)
% \end{equation}
%

Na equação \ref{eq_triplet_loss}, é a função \textit{Loss} utilizada para treinar o modelo. Onde, $x_i^a$ é
a imagem âncora, $x_i^p$ é a imagem positiva e $x_i^n$ é a imagem negativa, todas referentes a i-ésima
tripla e $m$ é um hiperparâmetro que define a margem entre o par positivo e par negativo.
Então, quanto mais parecidos forem os \textit{embeddings} da imagem âncora com a imagem positiva,
menor será a perda, sendo o inverso para a âncora com a imagem negativa.

\subsection{\textit{Triplet Distillation}}
Para realizar o treinamento dos modelos MobileNetV3 e Rafael-2, foi utilizada a técnica de
\textit{Knowledge Distillation} \cite{hinton2015distilling} com \textit{Triplet Loss}, conhecida como
\textit{triplet loss} \cite{triplet_distillation_face_recognition}.
Ela utiliza cálculo da \textit{Loss Function} (\ref{eq_triplet_loss}) como base, adicionando a distância entre os
\textit{embeddings} do modelo estudante e do modelo professor, como pode ser visto na fórmula
\ref{eq_triplet_distillation}.

\begin{equation}\label{eq_triplet_distillation}
	Loss = \frac 1 N \sum _i ^N max(D(x_i^a, x_i^p) - D(x_i^a, x_i^n) + d, 0))
\end{equation}

\begin{equation}\label{eq_triplet_loss_teacher_dist}
	d = max(T(x_i^a, x_i^n) - T(x_i^a, x_i^p), 0)
\end{equation}

Onde $T$ (\ref{eq_triplet_loss_teacher_dist}) é a distância entre os \textit{embeddings} do modelo estudante e
professor, e $d$ (\ref{eq_triplet_distillation}) é utilizado como uma margem dinâmica entre par positivo e negativo.

% TODO:
% - Citar modelos treinados (Triplet Distillation)
% 	- Rafael
% 	- MobileNetV3Small

\section{Método de avaliação}\label{sec_avaliacao_modelo}
Para avaliar o modelo, primeiro, as características de cada imagem e sua versão espelhada
(\textit{flip} horizontal) são extraídas pelo modelo. Depois é realizada a verificação da face,
calculando a distância de cosseno entre os vetores de características
\cite{triplet_distillation_face_recognition}, onde a imagem com menor distância é considerada a previsão do
modelo.

TODO: ADICIONAR IMAGEM
% TODO: DETALHAR

\section{\textit{Datasets}}\label{sec_datasets}
Nessa seção serão apresentados os dois \textit{datasets} utilizados, \textit{Labeled Faces in the Wild} (LFW),
que foi utilizado para o treinamento dos modelos, e Faces - UFS, que foi utilizado para a validação dos modelos.

\subsection{\textit{Labeled Faces in the Wild}}
Este \textit{dataset} possui imagens da face de várias celebridades, classificadas com o nome da pessoa.
Para esse trabalho, foi utilizada uma variação do LFW, que agrupa as imagens em triplas, \textbf{âncora},
\textbf{positiva} e \textbf{negativa}, para ser utilizado como dataset de treinamento, seguindo a técnica
de \textit{triplet distillation} \cite{triplet_distillation_face_recognition}.


Para realizar o treinamento do modelo, foi utilizada uma variação do \textit{dataset}
\textit{Labeled faces in the Wild} (LFW), que possui várias amostras contendo triplas com as imagens das faces,
assim facilitando o uso da técnica de \textit{triplet distillation} para o treinamento.

TODO: ADICIONAR DATASET DA UFS (LEANDRO)
\subsection{Faces - UFS}\label{subsec_dataset_faces_ufs}
Este \textit{dataset} possui faces coletadas de alunos da Universidade Federal de Sergipe (UFS) \cite{leandro}.
Para coletar essas imagens, foi disponibilizado um estande com uma câmera, monitor, teclado e mouse, permitindo
que qualquer pessoa pudesse salvar uma imagem com o seu nome.

Esse \textit{dataset} foi utilizado na etapa de validação do modelo (\ref{sec_avaliacao_modelo}), por apresentar faces
novas e com um padrão diferente do LFW, já que o seu público consiste em celebridades, principalmente dos Estados
Unidos. Enquanto o \textit{dataset} coletado na UFS possui um público diferente, estudantes ou funcionários da UFS.

\section{Resultados}\label{sec_resultados}
Para obter os resultados, primeiro foram definidas as métricas que serão usadas para avaliar os modelos. Em seguida,
cada modelo foi submetido ao método de avaliação escolhido (\autoref{sec_avaliacao_modelo}), com o objetivo de
validar e comparar o desempenho na tarefa de reconhecimento facial de cada um, junto com o seu tempo de execução.

Para realizar a avaliação na tarefa de reconhecimento facial, os modelos foram executados em um computador com um
AMD Ryzen 5 4600G. Na etapa de avaliação de performance em microcontroladores, foi utilizado o ESP32-S3 N16R8,
descrito na \autoref{sec_dispositivo}.

\subsection{Métricas usadas}\label{sect_restultados_metricas}
Para avaliar o desempenho, foram utilizadas as métricas de precisão (\ref{sect_restultados_metricas_precisao})
e acurácia (\ref{sect_restultados_metricas_acuracia}). Como a avaliação é binária, a imagem prevista é a imagem
verdadeira, também foi utilizada a matriz de confusão, para facilitar o entendimento dos acertos e erros, quanto
facilitar o cálculo da precisão e acurácia.

\label{sect_restultados_metricas_precisao}
A precisão pode ser calculada pela fórmula \ref{eq_precisao}. Onde a variável $TP$ indica o valor que foi previsto corretamente como verdadeiro,
enquanto a variável $FP$ indica o valor que foi previsto falsamente como verdadeiro.
Considerando o método de avaliação utilizada, a variável $TP$ indica os casos onde o modelo escolheu corretamente a
imagem, e a variável $FP$  indica quando o modelo escolheu a imagem errada.

\begin{equation}\label{eq_precisao}
	\text{Precisão} = \frac {TP} {TP + FP}
\end{equation}

\label{sect_restultados_metricas_acuracia}
Enquanto isso a acurácia pode ser calculada pela fórmula \
\begin{equation}\label{eq_acuracia}
	\text{Acurácia} = \frac {TP + TN} {TP + TN + FP + FN}
\end{equation}

\subsection{Avaliação do modelo}
Para a avaliação do modelo, o método descrito na \autoref{sec_avaliacao_modelo} foi executado 10 vezes,
utilizando o \textit{dataset} Faces - UFS (\ref{subsec_dataset_faces_ufs}), medindo o tempo médio de inferência
e acurácia de cada execução. Com isso esses resultados, foi gerada a \autoref{tempo_inferencia_modelos}, onde as
métricas são dos modelos descritos na \autoref{sec_modelo_utilizado}.


\begin{center}
\begin{table}[htb]
\centering
\ABNTEXfontereduzida
\caption[Acurácia e tempo de inferência (Faces - UFS)]{Acurácia e tempo de inferência (Faces - UFS)}
\label{tempo_inferencia_modelos}
\begin{tabular}{ |c|c|c|c| }
	\hline
	\textbf{Modelo} & \textbf{Acurácia} & \textbf{Tempo médio de inferência (ms)} & \textbf{Método de treinamento} \\
	\hline
	MobileFaceNet 		& 	100.00\% 	& $140 \pm 0.00266$ & Pré-treinado + \textit{Triplet Loss} \\
	Rafael-2	 	& 	 16.88\% 	& $120 \pm 0.00014$ & \textit{Triplet Distillation} \\
	MobileNetV3Small 	& 	 24.02\% 	& $114 \pm 0.00013$ & \textit{Triplet Distillation} \\
	\hline
\end{tabular}
\legend{Fonte: Autor}
\end{table}
\end{center}

Após a avaliação, foi possível perceber a diferença entre

DISCUTIR RESULTADOS
