\chapter{Resultados preliminares}

Neste capítulo serão apresentados testes feitos antes do projeto, eles tiveram a finalidade de exercitar o
conteúdo estudado durante o trabalho. O objetivo principal é usar as técnicas de compressão de modelos para
criar modelos menores e mais eficientes.

% TODO: Revisar

% Destilamento de conhecimento
\section{Destilamento de conhecimento (modelo Professor-Aluno)}
Para fazer o experimento com destilamento de conhecimento foi utilizada da base STL-10, que possui 500 imagens para
treinamento e 800 para teste, com resolução de $96 \times 96$ e 3 canais de cor (RGB). Como o conjunto de dados
não possui muitas imagens, foi aplicada a técnica de \textit{data augmentation} (aumento de dados) para reduzir o
\textit{overfitting}.

Como já foi descrito na \autoref{conceitos_destilamento}, o objetivo dessa etapa é utilizar o conhecimento do modelo
Professor(mais robusto e pré-treinado) para treinar o modelo Aluno (mais simples e sem pré-treinamento).
Onde o modelo professor (\autoref{res_professor}) é a ResNet-50  \cite{resnet} e o modelo estudante é gerado pelo
\autoref{res_aluno_1}.

\begin{codigo}[!htb]
    \caption{Criação do modelo Professor}
    \label{res_professor}
    \begin{lstlisting}[language = python]
	preprocess_input = tf.keras.applications.resnet50.preprocess_input
	base_model = tf.keras.applications.resnet.ResNet50(input_shape=IMG_SHAPE,
					   include_top=False,
					   pooling='avg',
					   weights='imagenet')
	base_model.trainable = False
	input = tf.keras.Input(shape=(96, 96, 3))
	x = input
	x = preprocess_input(x)
	x = base_model(x, training=False)
	x = tf.keras.layers.Dropout(0.2)(x)
	output = tf.keras.layers.Dense(n)(x)
	teacher = tf.keras.Model(input, output)
    \end{lstlisting}
    \legend{Fonte: Autor}
\end{codigo}

\begin{codigo}[!htb]
    \caption{Criação do modelo Aluno}
    \label{res_aluno_1}
    \begin{lstlisting}[language = python]
	def create_student_model():
		i = tf.keras.layers.Input(shape=IMG_SHAPE)
		x = add_cnorm_layer(32, i)
		x = add_cnorm_layer(64, x)
		x = add_cnorm_layer(128, x)
		x = tf.keras.layers.Flatten()(x)
		x = tf.keras.layers.Dropout(0.2)(x)
		x = tf.keras.layers.Dense(1024, activation='relu')(x)
		x = tf.keras.layers.Dropout(0.2)(x)
		x = tf.keras.layers.Dense(n)(x)
		return tf.keras.Model(i, x)

	def add_cnorm_layer(size, x):
		x = tf.keras.layers.Conv2D(size, (3, 3), padding='same', activation='relu')(x)
		x = tf.keras.layers.BatchNormalization()(x)
		x = tf.keras.layers.Conv2D(size, (3, 3), padding='same', activation='relu')(x)
		x = tf.keras.layers.BatchNormalization()(x)
		x = tf.keras.layers.MaxPooling2D((2, 2))(x)
		return x
    \end{lstlisting}
    \legend{Fonte: Autor}
\end{codigo}

Para aumentar a precisão do modelo Aluno com o destilamento de conhecimento, foi utilizada a otimização
\textit{bayesiana}, para procurar os valores dos hiperparâmetros $\alpha$ e \textit{Temperature}.
Os possíveis valores de $\alpha$ foram 0.1, 0.5, 0.01 e 0.25.
E os possíveis valores de \textit{Temperature} foram 2, 5, 7, 10, 12, 15, 17 e 20.

\begin{center}
\begin{table}[htb]
\ABNTEXfontereduzida
\caption[Acurácia dos modelos]{Acurácia dos modelos.}
\label{tab-nivinv}
\begin{tabular}{ |c|c|c|c|c| }
	\hline
	\textbf{Modelo} & \textbf{Com destilamento de conhecimento?}  & \textbf{Acurácia (validação)}
		   & \textbf{$\alpha$} & \textbf{\textit{Temperatura}} \\
	\hline
	ResNet-50 	& 	Não 	& 	90.65\%	& 	- 	& 	-	 \\
	Aluno-1 	& 	Não 	& 	76.80\%	& 	- 	& 	-	 \\
	Aluno-1 	& 	Sim 	& 	83.79\%	& 	0.1 	& 	5	 \\
	\hline
\end{tabular}
\legend{Fonte: Autor}
\end{table}
\end{center}

%TODO: Escrever sobre pruning
%TODO: Escrever sobre quantização
