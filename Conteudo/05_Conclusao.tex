\chapter{Conclusão}

Este trabalho teve como objetivo principal a compressão de uma rede neural artificial para viabilizar o
reconhecimento facial em dispositivos embarcados, buscando manter uma alta acurácia e baixa latência no
dispositivo, assim o tornado viável o seu uso em tarefas que executam em tempo real.
% , possibilitando que a mesma consiga auxiliar
% na tarefa de reconhecimento facial, mantendo uma alta acurácia e baixa latência em dispositivos embarcados.

Para alcançar esse objetivo, foi realizada a compressão, treinamento e avaliação de modelos de reconhecimento
facial, onde foi desenvolvidos \textit{benchmarks} para medir a acurácia e latência dos modelos, para
dispositivos de propósito geral e embarcados. Os dispositivos utilizados foram \textit{Desktop}, Raspberry
Model B e ESP32-S3, junto com os interpretadores TensorFlow, TFLite Runtime e ESP TFLite Micro.
% \section{Expectativas alcançadas}
% Durante o inicio do trabalho, foi estimado o treinamento, avaliação e implantação de um modelo de reconhecimento
% facial, e desenvolvimento de uma aplicação para avaliar a acurácia e latência do modelo dentro de um sistema
% embarcado. Onde essas tarefas foram feitas no \autoref{experimentos}, utilizando o \textit{Desktop}, Raspberry
% Model B e ESP32-S3 como dispositivos para execução do modelo, e \textit{TensorFlow}, \textit{TfLite Runtime} e
% ESP TFLite Micro, como interpretadores dos modelos.

% \section{Resultados}
Como foi possível observar nos experimentos, o melhor modelo para realizar o trabalho de reconhecimento
facial é o MobileFaceNet com quantização para o tipo uint8, pois ele apresenta um acurácia alta, mesmo possuindo
um tempo de resposta elevado. Considerando isso, esse modelo não é o mais indicado para realizar reconhecimento
facial em tempo real, pois ele demora 3.7 segundos gerando os \textit{embeddings} de uma imagem, ou seja, ele
é muito lento para ser executado em tempo real, visto que o ideal seria alcançar uma média de 33.33
milissegundos, para alcançar 30 \textit{frames} por segundo.

Inicialmente, esperava-se que fosse possível realizar compressão ao ponto que o modelo MobileFaceNet executasse
a tarefa de reconhecimento facial com baixa latência, ou que seria possível aplicar
\textit{Triplet Distillation} no modelo Rafael-2 para melhorar a sua acurácia consideravelmente, a tornando
próxima do MobileFaceNet.
Porém, como é possível observar, esse resultado não foi alcançado, uma vez que o único modelo com acurácia alta
foi o MobileFaceNet, que possui um tempo de inferência elevado no ESP32-S3, o tornando inviável para a
executá-lo em tempo real.

\section{Trabalhos futuros}
Para os trabalhos futuros, sugere-se que seja feita a expansão na  quantidade de hardware testados, com o objetivo de
avaliar a latência e acurácia dos modelos utilizados em outros dispositivos. Servindo como base, outros modelos do
Raspberry Pi ou ESP32, ou plataformas de outras fabricantes, como o Jetson Nano da NVidia.
Outra sugestão seria desenvolver um produto ou demonstração funcional que integre a detecção e reconhecimento facial em
aplicações práticas, usando como ponto de partida o \textit{benchmark} desenvolvido para o ESP32-S3.
O objetivo principal das sugestões feitas é expandir o experimento feito, buscando um ambiente onde a tarefa de
reconhecimento facial seja executada em um dispositivos embarcado de forma eficiente, mantendo uma baixa latência e
alta acurácia, assim a tornando mais acessível e fácil de aplicar em diversas situações.

% Considerando isso, sugere-se que, como trabalhos futuros, seja feita um estudo utilizando diferentes hardwares,
% com o objetivo de aumentar a amostra de dispositivos testados e encontrar um dispositivo com baixo custo que
% seja capaz de realizar a tarefa de reconhecimento facial em tempo real. Outra sugestão seria utilizar
% \textit{benchmark} construído para o ESP32-S3 como ponto de partida para desenvolver um produto ou demonstração
% funcional que integre a detecção e reconhecimento facial em aplicações práticas.
