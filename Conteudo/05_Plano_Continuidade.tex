\chapter{Planos de continuidade}
No Trabalho de Conclusão de Curso 1, foram realizadas as etapas de aprendizado e resultados de experimentos
relacionados ao tema.

\begin{center}
\begin{table}[htb]
\centering
\ABNTEXfontereduzida
\caption[Cronograma das atividades]{Cronograma das atividades.}
\label{tabela_plano}
\begin{tabular}{ |l|c|c|c|c|c| }
	\hline
	Atividade & Maio & Jun. & Jul. & Ago. & Set. \\
	\hline
	Busca de soluções existentes & X & & & & \\
	\hline
	Escolha de um modelo base & X & &  & & \\
	\hline
	Aplicação de técnicas de compressão e execução de experimentos & & X & X & & \\
	\hline
	Proposição de uma solução e teste & & & & X & X \\
	\hline
\end{tabular}
\legend{Fonte: Autor}
\end{table}
\end{center}

Na primeira etapa, será realizada a busca de modelos de reconhecimento facial, com o foco em sistemas embarcados,
onde um o ponto de partida será o trabalho "Edge FR - Desenvolvimento de um modelo de reconhecimento facial para dispositivos
embarcados de baixo custo"\  \cite{leandro}.
Depois disso, esses modelos serão comparados, tendo como principais métricas acurácia e pegada de memória.
Com o modelo base escolhido, serão utilizadas técnicas de compressão, com o objetivo de produzir um modelo que seja
computacionalmente barato e consuma pouca memória.
Para finalizar, o modelo final será testado em diversos dispositivos embarcados, onde o objetivo principal é que ele
seja executado em dispositivos baratos, como pouco poder computacional e memória disponível.

Vale ressaltar que, o cronograma é uma expectativa das atividades que serão realizadas durante o Trabalho de Conclusão
de Curso 2, algumas atividades podem ser prolongadas, reduzidas, adicionadas ou removidas.
