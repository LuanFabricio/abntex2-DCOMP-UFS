\begin{apendicesenv}

% Imprime uma página indicando o início dos apêndices
\partapendices

% ----------------------------------------------------------
\chapter{Modelo Professor}\label{apendice_professor}
% ----------------------------------------------------------

Código utilizado para criar a o modelo Professor usando a ResNet-50 \cite{resnet} com TensorFlow 2.0.

\begin{codigo}[!htb]
    \caption{Criação do modelo Professor}
    \label{res_professor}
    \begin{lstlisting}[language = python]
	preprocess_input = tf.keras.applications.resnet50.preprocess_input
	base_model = tf.keras.applications.resnet.ResNet50(input_shape=IMG_SHAPE,
					   include_top=False,
					   pooling='avg',
					   weights='imagenet')
	base_model.trainable = False
	input = tf.keras.Input(shape=(96, 96, 3))
	x = input
	x = preprocess_input(x)
	x = base_model(x, training=False)
	x = tf.keras.layers.Dropout(0.2)(x)
	output = tf.keras.layers.Dense(n)(x)
	teacher = tf.keras.Model(input, output)
    \end{lstlisting}
    \legend{Fonte: Autor}
\end{codigo}

% ----------------------------------------------------------
\chapter{Modelo Aluno}\label{apendice_aluno}
% ----------------------------------------------------------
Código utilizado para criar a o modelo Aluno com TensorFlow 2.0.

\begin{codigo}[!htb]
    \caption{Criação do modelo Aluno}
    \label{res_aluno_1}
    \begin{lstlisting}[language = python]
	def create_student_model():
		i = tf.keras.layers.Input(shape=IMG_SHAPE)
		x = add_cnorm_layer(32, i)
		x = add_cnorm_layer(64, x)
		x = add_cnorm_layer(128, x)
		x = tf.keras.layers.Flatten()(x)
		x = tf.keras.layers.Dropout(0.2)(x)
		x = tf.keras.layers.Dense(1024, activation='relu')(x)
		x = tf.keras.layers.Dropout(0.2)(x)
		x = tf.keras.layers.Dense(n)(x)
		return tf.keras.Model(i, x)

	def add_cnorm_layer(size, x):
		x = tf.keras.layers.Conv2D(size, (3, 3), padding='same', activation='relu')(x)
		x = tf.keras.layers.BatchNormalization()(x)
		x = tf.keras.layers.Conv2D(size, (3, 3), padding='same', activation='relu')(x)
		x = tf.keras.layers.BatchNormalization()(x)
		x = tf.keras.layers.MaxPooling2D((2, 2))(x)
		return x
    \end{lstlisting}
    \legend{Fonte: Autor}
\end{codigo}


\end{apendicesenv}
