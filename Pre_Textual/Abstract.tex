% resumo em inglês
\setlength{\absparsep}{18pt} % ajusta o espaçamento dos parágrafos do resumo
\begin{resumo}[Abstract]
 \begin{otherlanguage*}{english}
   Convolutional Neural Networks are becoming increasingly popular for solving various challenges, one of them being
   facial recognition, which is one of the tasks that this approach overcomes humans.
   However, this method usually requires a high computational power and amount of memory, which ends up limiting its
   use case in edge computing for embedded devices.
   This work focuses on addressing this issue, compressing a facial recognition model so that it is embedded and can
   perform operations at the edge, maintaining high accuracy and low response time.
   For this objective to be achieved, it will be necessary to use a model as a basis, so that it can be compressed and
   evaluated, where it will be chosen based on the comparison of the existing and used
   solutions.
   % TODO: traduzir essa parte
   To finish, this model will be embedded to perform operations on edge, both inference and processing on the device,
   from there, its effectiveness will be measured and compared with other solutions already created.

   \vspace{\onelineskip}

   \noindent
   \textbf{Keywords}: CNN. Model compression. Computer Vision. Embedded systems. Edge computing. Facial recognition.
 \end{otherlanguage*}
\end{resumo}
