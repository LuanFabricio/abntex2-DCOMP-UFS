% resumo em inglês
\setlength{\absparsep}{18pt} % ajusta o espaçamento dos parágrafos do resumo
\begin{resumo}[Abstract]
 \begin{otherlanguage*}{english}
   Convolutional Neural Networks are becoming increasingly popular for solving various challenges, one of them being facial
   recognition, which is one of the tasks where this approach is already on par with human performance.
   However, this method often requires high processing power and memory, which limits its use in edge computing scenarios
   with embedded devices.
   This work focuses on addressing this problem by compressing a facial recognition model so that it can be embedded and
   perform operations at the edge, maintaining high accuracy and low response time.
   To achieve this goal, it will be necessary to use a base model to be compressed and evaluated, which will be chosen by
   comparing existing and utilized solutions. The embedded model will be evaluated based on metrics such as accuracy,
   latency and memory occupation, and compared to other existing solutions.

   \vspace{\onelineskip}

   \noindent
   \textbf{Keywords}: CNN. Model compression. Computer Vision. Embedded systems. Edge computing. Facial recognition.
 \end{otherlanguage*}
\end{resumo}
