% resumo em português
\setlength{\absparsep}{18pt} % ajusta o espaçamento dos parágrafos do resumo
\begin{resumo}

Redes Neurais Convolucionais estão ficando cada vez mais populares para solução de diversos desafios, sendo um deles
o de reconhecimento facial, que é uma das tarefas onde essa abordagem já supera o ser humano.
Entretanto, esse método costuma exigir um alto poder de processamento e quantidade de memória, o que acaba
limitando o seu uso em casos de computação de borda com dispositivos embarcados.
Este trabalho tem como foco tratar esse problema, comprimindo um modelo de reconhecimento facial para que ele seja
embarcado e consiga realizar operações na borda, mantendo a acurácia alta e tempo de resposta baixo.
Para que esse objetivo seja cumprido, será necessário utilizar um modelo como base, para que ele seja comprimido e
avaliado, onde ele será escolhido a partir da comparação de soluções já existentes e utilizadas.
O modelo embarcado será avaliado com base em métricas como acurácia, F1 score, latência e ocupação de memória e
comparado a outras soluções existentes.

 \textbf{Palavras-chave}: CNN, Compressão de modelos, Visão computacional, Sistemas embarcados, Computação em borda,
 Reconhecimento facial
\end{resumo}
